\documentclass{../TechDoc}

\title{Анализ поведения временных систем с помощью динамических метрических графов.}
\author{Студент группы БПИ172}{А. А. Измайлов}
\academicTeacher{доцент департамента программной инженерии
	факультета компьютерных наук}{Л. В. Дворянский}

\documentTitle{Техническое задание}
\documentCode{RU.17701729.04.01-01 ТЗ 01-1}


\addto\captionsrussian{\def\refname{Список используемых источников}}

\begin{document}
	\maketitle
	\tableofcontents
	
	\section{Введение}
	\subsection{Наименование программы}
	\subsubsection{Наименование на русском языке}
	Анализ поведения временных систем с помощью динамических метрических графов.
	\subsubsection{Наименование на английском языке}
	Behaviour analysis of real time systems via dynamic metric graphs.
	\subsection{Краткая характеристика области применения}
	На момент создания приложения нет ни одной программы позволяющей перевести временную сеть Петри в метрический граф и обратно, так как эта область только развивается. Данная программа нацелена на масимальное покрытие всех различных конфигураций временных сетей Петри и метрических графов.
	\subsection{Основание для разработки}
	Приказ декана факультета компьютерных наук Национального исследовательского университета «Высшая школа экономики» № ХХХХХХХ от ХХ.ХХ.2019 "ХХХХХХХХХ«Об утверждении тем, руководителей курсовых работ студентов образовательной программы Программная инженерия факультета компьютерных наук».
	
	\section{Назначение программы}
	\subsection{Функциональное назначение}
	Программа представляет из себя транслятор, который принимает на вход временную сеть Петри\cite{PetriNet}, а на выход выдает метрический граф\cite{MGraphs} или сообщение о том, что конвертация не возможна, а также наоборот. Программа не всегда может перевести сеть Петри в метрический граф, так как у этих моделей разная выразительность.
	\subsection{Эксплуатационное назначение}
	Данная программа может быть использована при исследовании свойств сетей Петри применительно к метрическим графам или наоборот. Такое может быть полезно, так как по отдельности эти две модели хорошо изучены, но методы исследования одной модели никто не применял к исследованию второй модели.
	
	\section{Требования у программе}
	\subsection{Требования к составу выполняемых функций}
	Программа дожна:
	\begin{itemize}
		\item уметь считывать временную сеть Петри с внешнего хранилища;
		\item уметь считывать метрический граф с внешнего хранилища;
		\item конвертировать временную сеть Петри в метрический граф;
		\item предоставлять отчет о том, почему нельзя перевести сеть Петри в метрический граф;
		\item конвертировать метрический граф во временную сеть Петри;
	\end{itemize}

	Программа должна быть реализованна на языке платформы JVM.
	
	\subsection{Требования к интерфейсу}
	Программа должна иметь интерфейс для работы в терминале. То есть вся работа с программой может происходить через передачу параметров программе в командной строке.
	
	\subsection{Требования к входным и выходным данным}
	Формат входных и выходных данных для сетей Петри должен быть PNML\cite{PNML}. Для входны и выходных данных для метрических графов должен быть текстовый человекочитаемый формат, например XML\cite{XML} или JSON\cite{JSON}.
	
	\subsection{Условия эксплуатации}
	\subsubsection{Климатические условия}
	Климатические условия совпадают с климатическими условиями эксплуатации устройства,
	на котором исполняется программа.
	
	\subsubsection{Требования к квалицифакции оператора}
	Оператор должен:
	\begin{itemize}
		\item иметь понимание временных сетей Петри;
		\item иметь понимание метрических графов;
		\item уметь работать с командной строкой/терминалом;
		\item понимать устройство формата XML;
	\end{itemize}

	\subsection{Требования к составу и параметрам технических средств}
	Минимальные требования:
	\begin{itemize}
		\item Процессор архитектуры AMD или Intel с частотой не менее 2 266 МГц;
		\item Не менее 128 МБ ОЗУ;
		\item Не менее 200МБ на жестком диске;
		\item Клавиатура;
	\end{itemize}

	\subsection{Требования к информационной и программной совместимости}
	\begin{itemize}
		\item Одна из ниже перечисленных операционных систем\cite{javareq}:
		\begin{itemize}
			\item Windows 10
			\item Windows 8.x (настольная версия)
			\item Windows 7 с пакетом обновления 1 (SP1)
			\item Windows Vista SP2
			\item Windows Server 2008 R2 с пакетом обновления 1 (SP1) (64-разрядная версия)
			\item Windows Server 2012 и 2012 R2 (64-разрядная версия)
			\item Mac OS X 10.8.3+, 10.9+
			\item Oracle Linux 5.5+1
			\item Oracle Linux 6.x (32-разрядная версия), 6.x (64-разрядная версия)2
			\item Oracle Linux 7.x (64-разрядная версия)2
			\item Red Hat Enterprise Linux 5.5+1, 6.x (32-разрядная версия), 6.x (64-разрядная версия)2
			\item Red Hat Enterprise Linux 7.x (64-разрядная версия)2
			\item Suse Linux Enterprise Server 10 SP2+, 11.x
			\item Suse Linux Enterprise Server 12.x (64-разрядная версия)2
			\item Ubuntu Linux 12.04 LTS, 13.x, 14.x, 15.x, 16.x, 18.x
		\end{itemize}
		\item Установленная Java SE Runtime Environment 8\cite{Java} или выше
	\end{itemize}
	\subsection{Требования к программной документации}
	В рамках данной работы должна быть разработана следующая программная документация в соответствии и ГОСТ ЕСПД:
	\begin{itemize}
		\item «Программа для проверки поведенческих свойств сетей Петри с помощью редукций». Техническое задание\cite{TZbook};
		\item «Программа для проверки поведенческих свойств сетей Петри с помощью редукций». Программа и методика испытаний\cite{TESTbook};
		\item «Программа для проверки поведенческих свойств сетей Петри с помощью редукций». Текст программы\cite{SourceCodeBook};
		\item «Программа для проверки поведенческих свойств сетей Петри с помощью редукций». Пояснительная записка\cite{PZbook};
		\item «Программа для проверки поведенческих свойств сетей Петри с помощью редукций». Руководство оператора\cite{Operatorbook};
	\end{itemize}

	\section{Стадии и этапы разработки}       
	\subsection{Техническое задание}
	\begin{enumerate}
		\item Обоснование необходимости разработки 
		\begin{enumerate}
			\item Постановка задачи;
			\item Сбор теоретического материала;
			\item Выбор и обоснование критериев эффективности и качества разрабатываемого продукта;
		\end{enumerate}
		
		
		\item Научно-исследовательские работы
		\begin{enumerate}
			\item Определение структуры входных и выходных данных;
			\item Предварительный выбор методов решения поставленной задачи;
			\item Определение требований к техническим средствам;
			\item Обоснование возможности решения поставленной задачи.
		\end{enumerate}
		
		\item Разработка и утверждение технического задания
		\begin{enumerate}
			\item Определение требований к программе;
			\item Определение стадий, этапов и сроков разработки программы и документации на неё;
			\item Согласование и утверждение технического задания.
		\end{enumerate}
		
	\end{enumerate}
	
	\subsection{Рабочий проект}
	\begin{enumerate}
		\item Разработка программы
		\begin{enumerate}
			\item Реализация представления временной сети Петри в программе
			\item Реализация представления метрического графа в программе
			\item Реализация считывания сети Петри
			\item Реализация считывания метрического графа
			\item Реализация конвертации из сети Петри в метрический граф
			\item Реализация конвертации из метрического графа в сеть Петри
			\item Отладка программы.
		\end{enumerate}
		
		\item Разработка программной документации
		\begin{enumerate}
			\item Разработка программных документов в соответствии с требованиями ЕСПД.
		\end{enumerate}
		
		\item Испытания программы
		\begin{enumerate}
			\item Разработка, согласование и утверждение программы и методики испытаний;
			\item Проведение предварительных приемо-сдаточных испытаний;
			\item Корректировка программы и программной документации по результатам испытаний. 
		\end{enumerate}
	\end{enumerate}
	\subsection{Внедрение}
	\begin{enumerate}
		\item Подготовка и защита программного продукта
		\begin{enumerate}
			\item Подготовка программы и документации для защиты;
			\item Утверждение дня защиты программы;
			\item Презентация разработанного программного продукта;
			\item Передача программы и программной документации в архив НИУ ВШЭ.
		\end{enumerate}
	\end{enumerate} 

	\section{Технико-экономические показатели}
	В рамках данной работы расчет экономической эффективности не предусмотрен
	\subsection{Предполагаемая потребность}
	Данная программа может быть полезна при изучении схожести средст анализа метрических графов и временных сетей Петри. Также данную программу можно расширить, добавив к ней графический frontend. 
	
	\subsection{Экономические преимущества разработки по сравнению с отечественными и зарубежными аналогами}
	Данная программа не имеет аналогов, в силу сырости данной области.
	
	\section{Порядок контроля и приемки}
	Контроль и приемка разработки осуществляются в соответствии с программным документом «Анализ поведения временных систем с помощью динамических метрических графов». Программа и методика испытаний\cite{TESTbook};
	
	\newpage
	\bibliographystyle{utf8gost705u}  %% стилевой файл для оформления по ГОСТу
	\bibliography{biblio}     %% имя библиографической базы (bib-файла)
	\addtocounter{section}{2}
	\addcontentsline{toc}{section}{Список используемых источников}
	
	 \registrationList
\end{document}